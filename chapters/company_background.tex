\section{Company Background}

RAMLAB, based in Rotterdam, is a pioneer in the field of 3D metal printing, monitoring, and analysis solutions for Wire Arc Additive Manufacturing (WAAM). They are known for their high-quality WAAM systems and have developed MaxQ, a quality monitoring and control system for WAAM.

RAMLAB has three founding partners: Port of Rotterdam, InnovationQuarter and RDM Next, and collaborates with many other partners from academia, industry and government. RAMLAB conducts several research projects on WAAM, such as WAAMTOP, Grade2XL, AiM2XL and SMITZH.

RAMLAB's WAAM technology offers several advantages over conventional manufacturing processes. It allows for the manufacturing of large parts with dimensions over a cubic meter, offers additional design freedom, and uses a wide range of materials. It also offers the possibility of designing functionally graded components, where multiple materials can be combined to design a part.

They made headlines in 2017 by 3D printing a full-scale prototype of the world's first class approved ship's propeller. Their work continues to push the boundaries of what's possible in additive manufacturing - currently, they are working on manufacturing several-meter-wide pressure vessels for underwater structures, an unprecended feat in the field of additive manufacturing.


