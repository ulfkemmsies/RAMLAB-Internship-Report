\section*{Introduction}

This report is a recollection of the internship done, as required by the
Delft University of Technology, in pursuit of the MSc. in Space Engineering,
by the author at, the Rotterdam
Additive Manufacturing Laboratory, more commonly known as RAMLAB.

This internship, broadly speaking, was an investigation into the
application of Wire Arc Additive Manufacturing (WAAM) to the production of
metallic parts with magnetic properties.

WAAM is a production process used to 3D print or repair metal parts. It belongs to the Direct Energy Deposition (DED) family of Additive Manufacturing processes. The process is executed by depositing layers of metal on top of each other until a desired 3D shape is created. This is achieved by a welding robot integrated with a power source.

The report will start with a brief introduction to the company, followed by
a description of the problem that was investigated. The methods used to
investigate the problem will be described, followed by the outcome of the
investigation. The report will conclude with a summary of the conclusions
drawn from the investigation, and recommendations for future work.
