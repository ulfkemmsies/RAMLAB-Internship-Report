\section{Conclusions/Recommendations}

There are three main points to be made given the results of this investigation so far:

\begin{itemize}
    \item Vacoflux-17 can be printed with a relatively high quality and accuracy using MIG WAAM under certain conditions.
    \item There are "bubbling" and "humping" defects which occur consistently in single-bead walls after enough layers have been built up. These defects are not well understood, and persist even after mitigation attempts using 6-7 different strategies failed. It is surmised that these defects can be solved through process parameter optimization and yet unexplored modifications.
    \item Welding worsens the material's magnetic performance, but annealing in a H$_2$ atmosphere for 10 hours at 850 $^o$C corrects this loss, and even improves on the wire's baseline performance. More tests, like a permeameter, are needed to more fully characterize the material's magnetic performance in different regimes, like AC loading.
    \item The microstructure and compositional tests have not yet been completed, but will be included in this point when they soon are.
\end{itemize}